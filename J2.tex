\chapter{多维塑性}
本章将介绍多维情况下与$J_2$相关的屈服准则,$J_2$相关的屈服准则通常可以描述金属类材料的塑性应力应变关系。常用的与$J_2$相关的屈服准则有Tresca准则和Mises准则,并与Mises理想弹塑性屈服准则为例,介绍多维情况下切向弹塑性模量的推导过程。
\section{Lode坐标系}
对于各向同性材料,其屈服应力应当与方向无关。即任意一点处的任意一平面上的应力达到屈服应力时,该点处其余平面上的应力也达到屈服应力。所以为了方便描述,屈服函数可采用主应力$\sigma_i$表示
\begin{equation}
f(\sigma_1,\sigma_2,\sigma_3) = 0
\end{equation}
此时,屈服面可建立与由主应力$\sigma_i$所组成的坐标系中,此坐标系也成为\textbf{主应力空间}。但是采用主应力进行表示时,屈服面函数形式通常较为复杂,而是改为应力的三个不变量$I_1,I_2,I_3$表示
\begin{equation}
f(I_1,I_2,I_3) = 0
\end{equation}\par
在主应力空间上,将主应力进行谱分解可在主应力空间的基础上建立以不变量$I_i$为基础的坐标系。在主应力空间中的一点$\boldsymbol \sigma=\{\sigma_1,\sigma_2,\sigma_3\}$可采用相应的基向量$\boldsymbol e_i$和主应力表示为
\begin{equation}\label{ch_J2:sigma}
\boldsymbol \sigma = \sigma_1 \boldsymbol e_1 + \sigma_2 \boldsymbol e_2 + \sigma_3 \boldsymbol e_3
\end{equation}
对主应力引入谱分解,主应力分解为静水压力和主偏应力表示
\begin{equation}\label{ch_J2:decompose}
\sigma_i = p + s_i, \quad p = \frac{1}{3}(\sigma_1+\sigma_2+\sigma_3), \quad s_i = \sigma_i - p
\end{equation}
将式\eqref{ch_J2:decompose}代入式\eqref{ch_J2:sigma},主应力空间中的应力向量可分解为
\begin{equation}
    \begin{split}
        \boldsymbol \sigma &= \sum_{i=1}^3 \sigma_i \boldsymbol e_i = \sum_{i=1}^3 (p + s_i) \boldsymbol e_i \\
                           &= p \sum_{i=1}^3 \boldsymbol e_i + \sum_{i=1}^3 s_i \boldsymbol e_i = \frac{I_1}{\sqrt{3}} \boldsymbol e_z + \sum_{i=1}^3 s_i \boldsymbol e_i
    \end{split}
\end{equation}
其中,$\boldsymbol e_z = \frac{\sum_{i=1}^3\boldsymbol e_i}{\sqrt 3}$的方向为静水压力轴线方向,该向量模为$\sqrt{3}$。过坐标系原点做垂直于静水压力轴的平面称为$\pi$平面,由上式可知,所有$\sum_{i=3}^3 s_i \boldsymbol e_i$均平行于$\pi$平面,并且根据式\eqref{ch_stress:J22},该向量的模为
\begin{equation}
    \sqrt{\sum_{i=1}^3 s_i^2} = \sqrt{2J_2},\quad \sum_{i=1}^3 s_i \boldsymbol e_i = \sqrt{2J_2} \boldsymbol e_r
\end{equation}
以$\frac{I_1}{\sqrt 3}$为$z$轴,$\sqrt{2J_2}$为半径$r$轴,如果在$\pi$平面上选取一合适的夹角$\theta$,即可组成圆柱坐标系表示主应力空间中的应力状态。\par
令主应力空间基向量$\boldsymbol e_i$在$\pi$平面上的投影为$\boldsymbol e'_i$,$\boldsymbol e_i = \sqrt{\frac{2}{3}} \boldsymbol e'_i$。在$\pi$平面上,以主应力空间坐标原点为原点,$\boldsymbol e'_2$为$y$轴,重新建立一直角坐标系$xOy$。假设一点应力状态在$\pi$平面上的投影为$\sqrt{2J_2}\boldsymbol e_r$,此时可选取$\boldsymbol e_r$与$x$轴的夹角$\theta_\sigma$为圆柱坐标系的夹角变量,称为\textbf{Lode角}。直角坐标系$(x,y)$与极坐标系$(\sqrt{2J_2},\theta_\sigma)$具有如下关系
\begin{subequations}\label{ch_J2:xy}
\begin{align}
    x &= \frac{1}{\sqrt 2}(s_1-s_3) = \frac{1}{\sqrt 2}(\sigma_1-\sigma_3) = \sqrt{2J_2}\cos \theta_\sigma \\
    y &= \frac{1}{\sqrt 6}(2s_2-s_1-s_3) = \frac{1}{\sqrt 6}(2\sigma_2-\sigma_1-\sigma_3) = \sqrt{2J_2} \sin \theta_\sigma
\end{align}
\end{subequations}
根据上述关系可得Lode角为
\begin{equation}
    \theta_\sigma = \arctan \frac{y}{x} = \frac{1}{\sqrt 3} \mu_\sigma
\end{equation}
其中
\begin{equation}
\mu_\sigma = \frac{2s_2-s_1-s_3}{s_1-s_3} = \frac{3s_2}{s_1-s_3} = \frac{2\sigma_2-\sigma_1-\sigma_3}{\sigma_1-\sigma_3}
\end{equation}
式中,$\mu_\sigma$为\textbf{Lode参数},当处于单轴拉伸时$\mu_\sigma=-1$;当处于纯剪状态时$\mu_\sigma=0$;当处于单轴压缩时$\mu_\sigma=1$。由于$\sigma_1\ge\sigma_2\ge\sigma_3$,有
\begin{equation}
-1 \le \mu_\sigma \le 1, \quad - \frac{\pi}{6} \le \theta_\sigma \le \frac{\pi}{6}
\end{equation}\par
Lode角也可通过下式由$J_2$和$J_3$确定
\begin{equation}
\cos3\theta_\sigma = \frac{3\sqrt 3 J_3}{2 J_2^{\frac{3}{2}}}
\end{equation}\par
根据几何关系,主应力$\sigma_i$和主偏应力$s_i$可采用$I_1$、$J_2$和$\theta_\sigma$表示为
\begin{equation}
    \begin{split}
        \sigma_1 &= p+s_1 = \frac{I_1}{3} + \frac{2\sqrt{J_2}}{\sqrt 3}\sin(\theta_\sigma + \frac{2\pi}{3}) \\
        \sigma_2 &= p+s_2 = \frac{I_1}{3} + \frac{2\sqrt{J_2}}{\sqrt 3}\sin \theta_\sigma \\
        \sigma_3 &= p+s_3 = \frac{I_1}{3} + \frac{2\sqrt{J_2}}{\sqrt 3}\sin(\theta_\sigma - \frac{2\pi}{3})
    \end{split}
\end{equation} \par
圆柱坐标系$(I_1,J_2,\theta_\sigma)$称为\textbf{Lode坐标系}或\textbf{Haigh–Westergaard坐标系},通过该坐标系屈服条件就可表示为
\begin{equation}
    f(I_1,J_2,\theta_\sigma) = 0 \; \mathrm{or} \; f(I_1,J_2,J_3) = 0
\end{equation}
\section{$J_2$模型}
\subsection{Tresca准则}
在Tresca准则中假设当剪应力达到最大时,材料发生屈服。此时屈服准则可以表示为
\begin{equation}
    f(\boldsymbol \sigma) = \sigma_1-\sigma_3 - \sqrt{\frac{3}{2}}K
\end{equation}
式中$K$为屈服相关的材料系数,$\sigma_1$、$\sigma_3$分别为最大和最小主应力。当主应力$\sigma_i$不区分大小时,屈服函数也可改写为
\begin{equation}
    f(\boldsymbol \sigma) = \max(\vert \sigma_1 - \sigma_2 \vert, \vert \sigma_2 - \sigma_3 \vert, \vert \sigma_3 - \sigma_1 \vert) - \sqrt{\frac{3}{2}}K
\end{equation} \par
从Tresca准则的屈服面方程中可以看出,该准则与静水压力无关(式中不存在$I_1=\sigma_1+\sigma_2+\sigma_3$),屈服面始终垂直于$\pi$平面。将式\eqref{ch_J2:xy}引入将主应力采用$\pi$平面上的$x,y$坐标表示可得屈服面方程在$\pi$平面上投影的形状
\begin{subequations}
    \begin{align}
        \vert \sigma_1 - \sigma_2 \vert &= \Big \vert \frac{\sqrt 2}{2}x - \frac{\sqrt 6}{2}y \Big \vert \le \sqrt{\frac{3}{2}}K \Rightarrow \Big \vert \frac{\sqrt 3}{3} x - y \big \vert \le K \\
        \vert \sigma_2 - \sigma_3 \vert &= \Big \vert \frac{\sqrt 2}{2}x + \frac{\sqrt 6}{2}y \Big \vert \le \sqrt{\frac{3}{2}}K \Rightarrow \Big \vert \frac{\sqrt 3}{3} x + y \big \vert \le K \\
        \vert \sigma_3 - \sigma_1 \vert &= \vert \sqrt 2x \vert \le \sqrt{\frac{3}{2}}K \Rightarrow \vert x \vert \le \frac{\sqrt 3}{2}K 
    \end{align}
\end{subequations}
从上式可知,在$\pi$平面上Tresca屈服面为正六边形。由于Tresca屈服面并不连续,当采用Associative流动法则时,塑性应变方向为$\frac{\partial f}{\partial \boldsymbol \sigma}$。但不连续的Tresca屈服面在不连续处导数并不存在,使用起来不方便。\par
除了采用主应力表示Tresca准则屈服面方程外,还可以使用$J_2$和Lode角$\theta_\sigma$表示
\begin{equation}
    f(J_2,\theta_\sigma)=2\sqrt{J_2}\cos \theta_\sigma - K \le 0
\end{equation}

\subsection{Mises屈服准则}
Mises屈服准则要求偏应力的大小不能超过相对应的屈服应力,其屈服准则可表示为
\begin{equation}
    f(\boldsymbol \sigma) = \sqrt{2J_2} - K \le 0
\end{equation}
其中,$\sqrt{2J_2}$为偏应力的大小,根据$J_2$的表达式可改写为主应力$\sigma_i$或$\pi$平面直角坐标系$x,y$表示
\begin{equation}
    \sqrt{2J_2} = \sqrt{\frac{1}{3}((\sigma_1-\sigma_2)^2 + (\sigma_2-\sigma_3)^2 + (\sigma_3-\sigma_1)^2)} = \sqrt{x^2+y^2}
\end{equation}\par
相较于Tresca准则,Mises准则具有光滑的屈服面,当材料为Associative材料时,其塑性应变可假设为
\begin{equation}
    \dot{\boldsymbol \varepsilon}^p = \dot \gamma \frac{\partial f}{\partial \boldsymbol \sigma}
\end{equation}
式中$\dot \gamma$为塑性应变增量的大小,$\frac{\partial f}{\partial \boldsymbol \sigma}$为塑性应变增量的方向,其表达式为
\begin{equation}
    \begin{split}
        \frac{\partial f}{\partial \boldsymbol \sigma} &= \frac{\partial f}{\partial J_2} \frac{\partial J_2}{\partial \boldsymbol \sigma} \\
                                                       &= \frac{1}{\sqrt{2J_2}} \frac{\partial (\frac{1}{2} \boldsymbol s: \boldsymbol s)}{\partial \boldsymbol \sigma} = \frac{1}{\sqrt{2J_2}} \boldsymbol s: \frac{\partial \boldsymbol s}{\partial \boldsymbol \sigma} \\ 
                                                       &= \frac{\boldsymbol s}{\Vert \boldsymbol s \Vert} = \boldsymbol n_s
    \end{split}
\end{equation}
式中$\boldsymbol n_s$为偏应力$\boldsymbol s$的方向。\par
当材料进入塑性阶段时,根据一致性条件有
\begin{equation}
    \begin{split}
        \dot f &= \frac{\partial f}{\partial \boldsymbol \sigma} : \dot{\boldsymbol \sigma} = \boldsymbol n_s : \boldsymbol C : (\dot{\boldsymbol \varepsilon}-\dot{\boldsymbol \varepsilon}^p) \\
               &= \boldsymbol n_s : \boldsymbol C : (\dot{\boldsymbol \varepsilon}-\dot \gamma \boldsymbol n_s) \\
               &=0
    \end{split} \Rightarrow
    \dot \gamma = \frac{\boldsymbol n_s : \boldsymbol C : \dot{\boldsymbol \varepsilon}}{\boldsymbol n_s : \boldsymbol C : \boldsymbol n_s}
\end{equation}
此时本构关系为
\begin{equation}
    \begin{split}
        \dot{\boldsymbol \sigma} &= \boldsymbol C : (\dot{\boldsymbol \varepsilon} - \dot{\boldsymbol \varepsilon}^p) \\
                                 &= \boldsymbol C : (\dot{\boldsymbol \varepsilon} - \frac{\boldsymbol n_s : \boldsymbol C : \dot{\boldsymbol \varepsilon}}{\boldsymbol n_s : \boldsymbol C : \boldsymbol n_s} \boldsymbol n_s) \\
                                 &= (\boldsymbol C - \boldsymbol n_s \otimes \boldsymbol n_s : \boldsymbol C) : \dot{\boldsymbol \varepsilon} \\
                                 &= \boldsymbol C^T : \dot{\boldsymbol \varepsilon}
    \end{split}
\end{equation}
其中,$\boldsymbol C^T$为四阶切向弹塑性张量。与一维情况不同,二维情况下的切向弹塑性张量并不为零。
