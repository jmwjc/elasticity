\chapter{绪论}\label{introduction}

本章前半部分主要介绍弹塑性力学课程的主要学习内容,通过一维拉压杆问题介绍课程的知识体系架构。本章后半部分介绍课程所需的数学工具——张量运算的符号定义和基本运算规则。

\section{拉压杆问题}
\subsection{平衡微分方程}
考虑如图所示的杆结构,该杆长度为$L$,横截面积为$A$。杆的左端固定,右端施加大小为$P$的拉力,杆整体施加单位长度的体力$b$。取一微元体$\Delta x$进行受力分析,微元体的左右两端收到内力(轴力)$F_N$的作用。$F_N$以拉为正、压为负,且$F_N$是关于$x$的函数。此时,微元体左端的轴力大小可表示为$F_N(x)$,方向为$x$轴负方向;微元体右端轴力的大小为$F_N(x+\Delta x)$,方向为$x$轴正方向。由$x$方向的平衡方程有
\begin{equation} \label{ch_introduction:xeq}
F_N(x+\Delta x) - F_N(x) + b \Delta x = 0
\end{equation}
根据极限的定义有
\begin{equation} \label{ch_introduction:lim}
    \lim_{\Delta x \rightarrow 0} \frac{f(x+\Delta x) - f(x)}{\Delta x} = \frac{df}{dx} = f_{,x}
\end{equation}
平衡方程\eqref{ch_introduction:xeq}中,当$\Delta x \rightarrow 0$,根据关系式\eqref{ch_introduction:lim}可得\textbf{平衡微分方程}
\begin{equation}
    F_{N,x} + b = 0, \quad 0 < x < L
\end{equation} \par
平衡微分方程描述杆件内部内力和体力之间的关系,杆件内部每一点处均满足该方程。\par

\subsection{几何方程}
接下来准备采用位移$u$描述杆件的变形,为了描述杆件内部点$x$处的变形如图所示,同样取微元体$\Delta x$进行分析,经过外力作用下,微元体移动且变形为微元体$\Delta x'$。此时,微元体左端的位移为$u(x)$,右端位移即为$u(x+\Delta x)$。根据几何关系有
\begin{equation}
    \Delta x + u(x+\Delta x) = u(x) + \Delta x' \; \Rightarrow \; \Delta x' - \Delta x = u(x+\Delta x) - u(x)
\end{equation}
变形则通过应变$\varepsilon$进行衡量,在一维情况下应变为单位长度的变形量,引入极限定义式\eqref{ch_introduction:lim}有
\begin{equation}\label{ch_introduction:kin}
    \varepsilon = \frac{\Delta x' - \Delta x}{\Delta x} = \frac{u(x+\Delta x) - u(x)}{\Delta x} \approx u_{,x}
\end{equation}
上式称为\textbf{几何方程},几何方程描述了位移和变形(应变)之间的关系。\par
值得注意的是,几何方程\eqref{ch_introduction:kin}满足线性函数的定义,即函数$f$满足下列关系式
\begin{equation}
\begin{cases}
f(x+y) = f(x) + f(y) \\ f(\alpha x) = \alpha f(x)
\end{cases}
\end{equation}
称$f$为线性函数。显然式\eqref{ch_introduction:kin}满足上述关系。除了线性的几何方程外,某些弹性力学问题的几何方程为非线性函数,如大变形问题,其几何方程为
\begin{equation}
    \varepsilon = u_{,x} + \frac{1}{2} u_{,x}^{2}
\end{equation} \par

\subsection{物理方程}
\textbf{物理方程}描述应力与应变直接的关系,应力指单位面积下所收到内力的大小,具体可参见后续章节。在一维情况下,应力$\sigma$为
\begin{equation}
    \sigma = \frac{F_N}{A}
\end{equation}
其中,$A$为横截面积。\par
应力应变之间的关系一般有弹性和塑性两种,弹性为外力卸载后,物体可恢复为原来的形状;而塑性情况下,外力去除后会产生不可恢复的变形。如图所示的应力应变关系图中,弹性卸载过程按原路径返回,而塑性情况下,应力卸载后能有一部分残留的应变,称该应变为塑性应变$\varepsilon^p$。\par
应力应变关系除了弹性、塑性区分外,还可以分为线性和非线性,如
\begin{equation}
\begin{cases}
    \sigma = E \varepsilon, & \text{Linear} \\
    \sigma = E \varepsilon^2, & \text{Non-linear}
\end{cases}
\end{equation}
其中,$E$为杨氏模量。\par

\subsection{边界条件}
边界条件主要分为两种,一种需要满足边界上的外力和内力之间的关系,称之为\textbf{应力边界条件或自然边界条件}$\Gamma^t$;另一种是约束位移的边界条件,称之为\textbf{位移边界条件或强制(本质)边界条件}$\Gamma^g$。两类边界条件满足$\Gamma^t \cap \Gamma^g = \varnothing,\; \Gamma^t \cup \Gamma^g = \Gamma$时,弹性力学问题存在唯一解。\par
应力边界条件同样可取边界处的微元体$\Delta x$进行分析,如图所示,微元体左端受轴力$F_N(x)$作用,右端受外力$P$作用,同时微元体内部受体力$b$作用,根据$x$方向平衡条件有
\begin{equation}
P - F_N(L) + b \Delta x = 0
\end{equation}
当$\Delta x \rightarrow 0$时
$$
F_N(L) = EAu_{,x}(L) = P
$$
位移边界条件相较应力边界条件要简单许多,本问题的位移边界条件为杆件左端固定,即
\begin{equation}
u(0) = 0
\end{equation}\par

\subsection{小节}
上述各方程即为求解弹性力学问题的基本方程,这些方程化综合简为\textbf{弹性力学方程组}。本节所示一维拉压杆问题的弹性力学方程组为
\begin{subequations}\label{ch_introduction:strong}
    \begin{align}
        &(EAu_{,x})_{,x} + b = 0, \quad 0<x<L\label{ch_introduction:eg1} \\
        &EAu_{,x}(L) = P\label{ch_introduction:eg2} \\
        &u(0) = 0\label{ch_introduction:eg3}
    \end{align}
\end{subequations}
假设体力$b$和拉力$P$均为常数,对式\eqref{ch_introduction:eg1}进行两次积分,积分过程中产生的待定系数通过式\eqref{ch_introduction:eg2}和\eqref{ch_introduction:eg3}确定,可得位移的解析表达式为
\begin{equation}\label{ch_introduction:ext}
    u(x) = - \frac{b}{2EA} x^2 + \frac{P+bL}{EA}x
\end{equation}
\par
弹性力学方程组是一类常见的偏微分方程组,各个方程确定了位移、应变、应力和边界条件(给定的位移和外力)之间的关系。每组关系之间均会存在线性关系和非线性关系,其中的非线性关系就可以相应的称为\textbf{几何非线性、材料非线性和边界条件非线性}。本教程主要学习先弹性部分内容和小部分塑性内容,这些内容遵循下列基本假定
\begin{itemize}
    \item \textbf{连续性}:位移$u$在全域可微(导数存在);
    \item \textbf{小变形}:几何方程为线性方程;
    \item \textbf{弹性}:可恢复性;
    \item \textbf{均匀性}:材料系数与位置无关;
    \item \textbf{各向同性}:材料系数与方向无关。
\end{itemize} \par
本节内容即为本教程所有内容的一个缩影,后续章节将会学习弹性力学方程组中各个方程是如何定义和得到的,并学习如何采用数值求解的方法(有限元法)求得相应的数值近似解。

\section{张量}
张量是非常实用且重要的数学工具,它可以简化方程的表达式和推导过程。本节将简要介绍张量的定义和一些常用的运算规则,其内容将有助于后续章节的学习。

\subsection{一阶张量}
张量与向量、矩阵非常类似,但又不完全一样。一阶张量与向量一样均,均为矢量(具有方向)。这里我们采用粗体表示矢量,在空间中的一点$\boldsymbol x$可用张量表示为
\begin{equation}\label{ch_introduction:coor}
\boldsymbol x = x_1 \boldsymbol e_1 + x_2 \boldsymbol e_2 + x_3 \boldsymbol e_3 = x_i \boldsymbol e_i
\end{equation}
其中,$x_i$称为张量$\boldsymbol x$的分量,下标$i=1,2,3$分别对应$x,y,z$方向。$\boldsymbol e_i$为$x_i$轴的基向量
\begin{equation}
\boldsymbol e_1 =
\begin{Bmatrix}
1\\0\\0
\end{Bmatrix}, \quad
\boldsymbol e_2 =
\begin{Bmatrix}
0\\1\\0
\end{Bmatrix}, \quad
\boldsymbol e_3 =
\begin{Bmatrix}
0\\0\\1
\end{Bmatrix}
\end{equation}\par
式\eqref{ch_introduction:coor}中的第二个等式是通过\textbf{Einstein求和公式}得到的,该求和公式要求表达式中出现两个相同的下标时,表达式需要对这个下标等于$1,2,3$情况进行求和。例如,$a_i,b_i$分别为一阶张量$\boldsymbol a, \boldsymbol b$的分量时,根据Einstein求和公式,下式成立。
\begin{equation}
a_1 b_1 + a_2 b_2 + a_3 b_3 = a_i b_i
\end{equation}\par
Einstein求和公式能将求和过程的表达式进行简化,在三维的弹性力学推导中有大大降低了公式的复杂程度。\par

\subsection{高阶张量}
高阶张量可通过一阶张量和并矢算子(Dyad product)得到,两个基向量的并矢过程可表示为
\begin{equation}
    \boldsymbol e_i \boldsymbol e_j \quad \mathrm{or} \quad \boldsymbol e_i \otimes \boldsymbol e_j
\end{equation}
这里需要注意,并矢得到的二阶基向量并没有明确的物理含义,在不同情况下的意义可能不一样,可能代表位置,或是代表方向。不是一般性,两个一阶张量$\boldsymbol u$、$\boldsymbol v$ 的并矢过程可表示为
\begin{equation}\label{ch_introduction:dyad}
    \begin{split}
        \boldsymbol u \otimes \boldsymbol v &= (u_i \boldsymbol e_i) \otimes (v_j \boldsymbol e_j) \\
                                            &= (u_1 \boldsymbol e_1 + u_2 \boldsymbol e_2 + u_3 \boldsymbol e_3) \otimes
                                               (v_1 \boldsymbol e_1 + v_2 \boldsymbol e_2 + v_3 \boldsymbol e_3) \\
                                            &= u_1 v_1 \boldsymbol e_1 \otimes \boldsymbol e_1
                                            +  u_1 v_2 \boldsymbol e_1 \otimes \boldsymbol e_2
                                            +  u_1 v_3 \boldsymbol e_1 \otimes \boldsymbol e_3 \\
                                            &+ u_2 v_1 \boldsymbol e_2 \otimes \boldsymbol e_1
                                            +  u_2 v_2 \boldsymbol e_2 \otimes \boldsymbol e_2
                                            +  u_2 v_3 \boldsymbol e_2 \otimes \boldsymbol e_3 \\
                                            &+ u_3 v_1 \boldsymbol e_3 \otimes \boldsymbol e_1
                                            +  u_3 v_2 \boldsymbol e_3 \otimes \boldsymbol e_2
                                            +  u_3 v_3 \boldsymbol e_3 \otimes \boldsymbol e_3 \\
                                            &= u_i v_j \boldsymbol e_i \otimes \boldsymbol e_j
    \end{split}
\end{equation}\par
从上式结果可以看出,通过Einstein求和公式表示时,式\eqref{dyad}的第一个等式可以跳过中间步骤得到最后的并矢结果,简化运算过程。 \par
本教程中常用的一阶张量主要有位移$\boldsymbol u$、外荷载$t$等。常用的高阶张量有
\begin{itemize}
    \item 应力 $\boldsymbol \sigma = \sigma_{ij} \boldsymbol e_i \otimes \boldsymbol e_j$ 包括9项分量;
    \item 应变 $\boldsymbol \varepsilon = \varepsilon_{ij} \boldsymbol e_i \otimes \boldsymbol e_j$ 包括9项分量;
    \item 四阶弹性张量 $\boldsymbol C = C_{ijkl} \boldsymbol e_i \otimes \boldsymbol e_j \otimes \boldsymbol e_k \otimes \boldsymbol e_l$ 包括81项分量。
\end{itemize}

\subsection{点乘与叉乘}
基向量的点乘(Dot product)定义为
\begin{equation}
    \boldsymbol e_i \cdot \boldsymbol e_j = \delta_{ij}
\end{equation}
其中,$\delta_{ij}$称为Kronecker delta 函数
\begin{equation}
    \delta_{ij} = \begin{cases}
        1 & i = j \\ 0 & i \ne j
    \end{cases}
\end{equation} \par
通过基函数点乘的结果可得到两个一阶张量的点乘结果为
\begin{equation}
    \begin{split}
        \boldsymbol u \cdot \boldsymbol v &= (u_i \boldsymbol e_i) \cdot (v_j \boldsymbol e_j) \\
                                          &= u_i v_j \boldsymbol e_i \cdot \boldsymbol e_j = u_i v_j \delta_{ij} \\
                                          &= u_i v_i
    \end{split}
\end{equation}
需要注意,根据Einstein求和公式,上式的结果式3项相加。\par
进一步将一个1阶张量和一个2阶张量点乘有
\begin{equation}
    \begin{split}
        \boldsymbol u \cdot \boldsymbol \sigma &= (u_i \tikzmark{ei}{$\boldsymbol e_i$}) \cdot (\sigma_{jk} \tikzmark{ej}{$\boldsymbol e_j$} \otimes \boldsymbol e_k) \\
                                                 &= u_i \sigma_{jk} \delta_{ij} \boldsymbol e_k \\
                                                 &= u_i \sigma_{ik} \boldsymbol e_k
    \end{split}
\end{equation}
\begin{tikzpicture}[overlay, remember picture]
    \draw[red,<->,ultra thick](ei) to [in=120,out=60] (ej);
\end{tikzpicture}
从上式中可以看出,点乘作用在相邻的两个基向量上,将其转化为 Kronecker delta 函数,整体张量阶次减小2阶,即由整体3阶减为1阶。\par
双点乘(Double dot product)算子是另一类常用的点乘算子,其运算规则与单点乘类似。例如,两个2阶张量$\boldsymbol \varepsilon$、$\boldsymbol \sigma$进行双点乘有
\begin{equation}
    \begin{split}
        \boldsymbol \varepsilon : \boldsymbol \sigma &= (\varepsilon_{ij} \tikzmark{e11}{$\boldsymbol e_i$} \otimes \tikzmark{e12}{$\boldsymbol e_j$}):(\sigma_{kl} \tikzmark{e13}{$\boldsymbol e_k$} \otimes \tikzmark{e14}{$\boldsymbol e_l$}) \\
                                                     &= \varepsilon_{ij} \sigma_{kl} \delta_{ik} \delta_{jl} \\
                                                     &= \varepsilon_{ij} \sigma_{ij}
    \end{split}
\end{equation}
\begin{tikzpicture}[overlay, remember picture]
    \draw[red,<->,ultra thick](e11) to [in=150,out=30] (e13);
    \draw[red,<->,ultra thick](e12) to [in=210,out=330] (e14);
\end{tikzpicture}
可以看出得到的结果中无基向量,而是个标量,总阶次由4阶变为0阶。\par
当一个4阶张量$\boldsymbol C$和一个2阶张量$\boldsymbol \varepsilon$进行双点乘可得
\begin{equation}
    \begin{split}
        \boldsymbol C : \boldsymbol \varepsilon &= (C_{ijkl} \boldsymbol e_i \otimes \boldsymbol e_j \otimes \tikzmark{e21}{$\boldsymbol e_k$} \otimes \tikzmark{e22}{$\boldsymbol e_l$}) : (\varepsilon_{mn} \tikzmark{e23}{$\boldsymbol e_m$} \otimes \tikzmark{e24}{$\boldsymbol e_n$}) \\
                                                &= C_{ijkl} \varepsilon_{mn} \delta_{km} \delta_{ln} \boldsymbol e_i \otimes \boldsymbol e_j \\
                                                &= C_{ijkl} \varepsilon_{kl} \boldsymbol e_i \otimes \boldsymbol e_j
    \end{split}
\end{equation}
\begin{tikzpicture}[overlay, remember picture]
    \draw[red,<->,ultra thick](e21) to [in=150,out=30] (e23);
    \draw[red,<->,ultra thick](e22) to [in=210,out=330] (e24);
\end{tikzpicture}
双点乘与单点乘类似,作用于相邻的两对基向量,整体张量阶次减小4阶。\par
两个基向量的叉乘(Cross product)算子可以表示为
\begin{equation}
    \boldsymbol e_i \times \boldsymbol e_j = \epsilon_{ijk} \boldsymbol e_k
\end{equation}
其中,$\epsilon_{ijk}$ 为枚举函数(Permutation notion)定义为
\begin{equation}
    \epsilon_{ijk} = \begin{cases}
         1 & ijk = 123,\;231,\;312 \\
        -1 & ijk = 321,\;213,\;132 \\
         0 & \mathrm{otherwise}
    \end{cases}
\end{equation}
两个1阶张量$\boldsymbol u, \boldsymbol v$进行叉乘有
\begin{equation}
    \begin{split}
        \boldsymbol u \times \boldsymbol v &= u_i v_j \boldsymbol e_i \times \boldsymbol e_j \\
                                           &= u_i v_j \epsilon_{ijk} \boldsymbol e_k \\
                                           &= (u_2 v_3 - u_3 v_2) \boldsymbol e_i \\
                                           &+ (u_3 v_1 - u_1 v_3) \boldsymbol e_j \\
                                           &+ (u_1 v_2 - u_2 v_1) \boldsymbol e_k
    \end{split}
\end{equation} \par
点乘和叉乘是除加减法外两类最基本的张量运算规则,两则具有一定的几何意义。两个1阶张量的点乘相当于一个张量向另一个张量投影的乘积,两个1阶张量的叉乘相当于这两个张量所组成的平行四边形的面积。

\subsection{张量微分}
张量的微分作用在张量的分量上,如位移$\boldsymbol u$对$x$的一阶导数可表示为
\begin{equation}
    \frac{\partial \boldsymbol u}{\partial x} = \frac{\partial u_i}{\partial x} \boldsymbol e_i = u_{i,1} \boldsymbol e_i
\end{equation}
其中,下标“$,1$“代表对$x$求导数。为了表示方便,这里定义微分张量(nabla tensor)为
\begin{equation}
    \nabla = \frac{\partial}{\partial x} \boldsymbol e_1 + \frac{\partial}{\partial y} \boldsymbol e_2 + \frac{\partial}{\partial z} \boldsymbol e_3 = \frac{\partial}{\partial x_i} \boldsymbol e_i
\end{equation}
运用张量的并矢、点乘和叉乘可以定义张量的梯度(Gradient)、散度(Divergence)、旋度(Curl):
\begin{itemize}
    \item \textbf{梯度(Gradient)}
        \begin{equation}
            \nabla \boldsymbol u = u_{i,j} \boldsymbol e_i \otimes \boldsymbol e_j
        \end{equation}
    \item \textbf{散度(Divergence)}
        \begin{equation}
            \nabla \cdot \boldsymbol u = u_{i,i} = u_{1,1} + u_{2,2} + u_{3,3}
        \end{equation}
    \item \textbf{旋度(Curl)}
        \begin{equation}
            \nabla \times \boldsymbol u = u_{i,j} \epsilon_{ijk} \boldsymbol e_k
        \end{equation}
\end{itemize}

\subsection{2阶张量的谱分解}\label{spectral}
对于一个2阶张量$\boldsymbol A$采用$3\times3$的方阵表示为
\begin{equation}
\boldsymbol A = \begin{bmatrix}
    A_{11} & A_{12} & A_{13} \\
    A_{21} & A_{22} & A_{23} \\
    A_{31} & A_{32} & A_{33}
\end{bmatrix}
\end{equation}
该方阵的特征值可通过下列特征方程求得:
\begin{equation}
    \det(\boldsymbol A - \lambda \boldsymbol 1) = - \lambda^3 + I_1 \lambda^2 - I_2 \lambda + I_3 = 0
\end{equation}
其中
\begin{align}
    \boldsymbol 1 &= \delta_{ij} \boldsymbol e_i \otimes \boldsymbol e_j \\
    I_1 &= \mathrm{tr} \boldsymbol A = A_{ii} = \lambda^{(1)} + \lambda^{(2)} + \lambda^{(3)}\label{ch_introduction:invarant1} \\
    \begin{split}\label{ch_introduction:invarant2}
    I_2 &= \frac{1}{2}(\mathrm{tr}^2 \boldsymbol A - \mathrm{tr}(\boldsymbol A \cdot \boldsymbol A)) \\
        &= \frac{1}{2}(A_{ii} A_{jj} - A_{ij} A_{ji}) \\
        &= \lambda^{(1)} \lambda^{(2)} + \lambda^{(2)} \lambda^{(3)} + \lambda^{(1)} \lambda^{(3)}
    \end{split} \\
    I_3 &= \det \boldsymbol A = \epsilon_{ijk} A_{1i} A_{2j} A_{3k} = \lambda^{(1)} \lambda^{(2)} \lambda^{(3)}\label{ch_introduction:invarant3}
\end{align}
式中$\boldsymbol 1$为2阶单位张量(Second order identity tensor),$I_1, I_2, I_3$分别为张量$\boldsymbol A$的第一、第二和第三不变量,当张量$\boldsymbol A$代表应力或应变时。坐标系进行旋转变换的情况下$I_1, I_2, I_3$保持不变,所以称它们为不变量。其中,$I_1$也称为$\boldsymbol A$的迹(Trace), $\mathrm{tr}\boldsymbol A = A_{ii}$。$\lambda^{(i)}, i=1,2,3$为$\boldsymbol A$的三个特征值。与之对应的三个特征向量$\boldsymbol n^{(i)}, i=1,2,3$满足
\begin{equation}
    (\boldsymbol A - \lambda^{(i)}\boldsymbol 1) \cdot \boldsymbol n^{(i)} = \boldsymbol 0
\end{equation}\par
上述过程称为2阶张量的特征分解,也称谱分解(Spectral decomposition)。除了谱分解外,对于2阶对称张量$\boldsymbol A$还可将其分为主方向部分(dilatational/volumetric)$\boldsymbol A^v$和偏方向部分(deviatoric)$\boldsymbol A^d$,采用分量可表示为
\begin{equation}
    \begin{split}
        A_{ij} &= \frac{1}{3} A_{kk} \delta_{ij} + (A_{ij} - \frac{1}{3} A_{kk} \delta_{ij}) \\
               &= A_{ij}^v + A_{ij}^d
    \end{split}
\end{equation}
式中
\begin{equation}\label{ch_introduction:decompose}
    \begin{cases}
        A_{ij}^v = \frac{1}{3} A_{kk} \delta_{ij} \\
        A_{ij}^d = A_{ij} - \frac{1}{3} A_{kk} \delta_{ij}
    \end{cases}
\end{equation}
从主方向部分和偏方向部分表达式可以看出,这两部分具有正交性,即
\begin{equation}
    \begin{split}
        \boldsymbol A^v : \boldsymbol A^d &= A_{ij}^v A_{ij}^d \\
                                          &= \frac{1}{3} A_{kk} \delta_{ij} (A_{ij} - \frac{1}{3} A_{ll} \delta_{ij}) \\
                                          &= \frac{1}{3} A_{kk} A_{ii}  - \frac{1}{9} A_{ll} \delta_{ii} \\
                                          &= 0
    \end{split}
\end{equation}\par
为了推导方便,这里可以引入4阶单位对称张量(Symmetirc fourth order identity tensor)$\boldsymbol I$和4阶单位偏张量(Deviatoric fourth order identity tensor)$\boldsymbol I^d$描述偏方向部分,4阶单位对称张量和偏张量具有如下表达式
\begin{align}
    \boldsymbol I &= \frac{1}{2}(\delta_{ik} \delta_{jl} + \delta_{il} \delta_{jk}) \boldsymbol e_i \otimes \boldsymbol e_j \otimes \boldsymbol e_k \otimes \boldsymbol e_l \\
    \begin{split}
        \boldsymbol I^d &= (\frac{1}{2}(\delta_{ik} \delta_{jl} + \delta_{il} \delta_{jk})-\frac{1}{3}\delta_{ij}\delta_{kl}) \boldsymbol e_i \otimes \boldsymbol e_j \otimes \boldsymbol e_k \otimes \boldsymbol e_l \\
                        &= \boldsymbol I - \frac{1}{3} \boldsymbol 1 \otimes \boldsymbol 1
    \end{split}
\end{align}
对于2阶对称张量$\boldsymbol A$具有如下关系
\begin{equation}
    \boldsymbol A = \boldsymbol I : \boldsymbol A, \quad
    \boldsymbol A^d = \boldsymbol I^d : \boldsymbol A
\end{equation}\par
偏方向部分$\boldsymbol A^d$的三个不变量称为$J_1, J_2, J_3$,这三个不变量也常用在弹塑性应力-应变本构关系中,其表达式根据公式\eqref{ch_introduction:decompose}和\eqref{ch_introduction:invarant1}-\eqref{ch_introduction:invarant3}可得
\begin{align}
    J_1 &= A_{ii}^d = A_{ii} - \frac{1}{3} A_{kk} \delta_{ii} = 0 \\
    \begin{split}
        J_2 &= \frac{1}{2}(A_{ij}^dA_{ij}^d - A_{ii}^dA_{jj}^d) = \frac{1}{2} A_{ij}^dA_{ij}^d \\
            &= \frac{1}{6}((\lambda^{(1)}-\lambda^{(2)})^2 + (\lambda^{(1)}-\lambda^{(3)})^2 + (\lambda^{(2)}-\lambda^{(3)})^2)
    \end{split} \\
    J_3 &= \epsilon_{ijk} A_{1i}^d A_{2j}^d A_{3k}^d
\end{align}\par
相应的应力、应变主方向部分和偏方向部分的张量如下所示
\begin{itemize}
    \item \textbf{静水压力}(主应力):$p = \frac{1}{3}\sigma_{ii}$
    \item \textbf{偏应力}:$s_{ij} = \sigma_{ij} - \frac{1}{3}\sigma_{kk} \delta_{ij} = \sigma_{ij} - p \delta_{ij}$
    \item \textbf{平均应变或球应变}(主应变):$\varepsilon_m = \frac{1}{3}\varepsilon_{ii}$
    \item \textbf{偏应变}:$e_{ij} = \varepsilon_{ij} - \frac{1}{3}\varepsilon_{kk} \delta_{ij}$
\end{itemize}
