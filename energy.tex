\chapter{变分原理}
本章介绍弹性力学变分原理中最常用的最小势能原理,最小势能原理是有限元法的基础,本章的内容对后续有限元法的学习有重要帮助。本章将从一维杆问题出发,详细讲述变分原理的推导过程和解题思路。
\section{最小势能原理}
最小势能原理要求物体的位移(即变形)应满足整体势能能量在已知边界条件下取最小值。以第\ref{introduction}章一维拉压杆问题为例,势能包括外力做的功和内力做的功。杆右端的拉力$P$所做的功为$Pu\vert_{x=L}$;杆件内部微元体$dx$上体力$b$所做功为$ubdx$。整体杆件外力所做的功为
\begin{equation}\label{ch_energy:ext}
    W^{ext}(u) = P u\vert_{x=L} + \int_{0}^{L} u b dx
\end{equation}\par
对于内力功,杆件的轴力与位移(应变)具有一定联系。如图所示,在线性假设下,轴力与应变呈线性关系。此时,微元体$dx$上轴力做功为
\begin{equation}
\begin{split}
    dW^{int}(u) &= \int_0^\varepsilon F_N d\varepsilon = \int_0^\varepsilon EA\varepsilon d\varepsilon \\
                &=\frac{1}{2} EA \varepsilon^2 = \frac{1}{2} F_N \varepsilon
\end{split}
\end{equation}
而整体内力功即为
\begin{equation}\label{ch_energy:int}
W^{int}(u) = \int_0^L \frac{1}{2} F_N(u) \varepsilon(u) dx =  \int_0^L \frac{1}{2} EA u_{,x}^2 dx
\end{equation}\par
结合内能\eqref{ch_energy:int}和外能\eqref{ch_energy:ext},杆件整体势能$\Pi$为
\begin{equation}
\begin{split}
    \Pi(u) &= W^{int}(u) - W^{ext}(u) \\
           &= \int_0^L \frac{1}{2}F_N\varepsilon dx - P u\vert_{x=L} - \int_{0}^{L} u b dx
\end{split}
\end{equation}
式中势能函数的变量为位移$u$,而位移$u$是关于$x$的函数,我们把这类以函数作为变量的函数称之为\textbf{泛函}(Functional),与之相对应的微分求导过程称为\textbf{变分}(Variation)。根据最小势能原理,真实位移应为势能泛函$\Pi$取最小时的位移。令势能泛函对变量$u$求一阶变分为零,可得势能泛函取得极值时的位移,即
\begin{subequations}\label{ch_energy:weak}
\begin{align}
    \delta \Pi(u) = &\int_0^L \delta u_{,x} EA u_{,x} dx - \delta u P \vert_{x=L} - \int_0^L \delta u b dx = 0 \\
    \Rightarrow &\int_0^L \delta u_{,x} EA u_{,x} dx = \delta u P \vert_{x=L} + \int_0^L \delta u b dx
\end{align}
\end{subequations}
式中$\delta u$为位移的变分,上式即是强形式\eqref{ch_introduction:strong}的弱形式,该弱形式不同于强形式,仅要求在全域满足上式即可。同时,上式也称为虚功原理,而$\delta u$即为虚功原理中的虚位移。虚位移$\delta u$与位移$u$两者并不相同,$\delta u$是物体所有可能发生的位移,所以称之为虚位移。需要注意的是,当$\delta \Pi(u)=0$时,势能泛函$\Pi(u)$不仅为极小值,同时也能证明是最小值,具体证明过程可参考相关文献。

\section{里兹-伽辽金法}
根据最小势能原理,弹性力学方程组就可转化为如下\textbf{里兹-伽辽金问题}(Ritz-Galerkin problem):对于求解空间中的任意函数$\delta u$,从求解空间中寻找位移$u$使得下式恒成立
\begin{equation}\label{ch_energy:galerkin}
    \int_0^L \delta u_{,x} EA u_{,x} dx = \delta uP\vert_{x\L} + \int_0^L \delta ub dx
\end{equation}
其中求解空间的选择就极为重要,在第\ref{introduction}章中已经知道,当体力$b$与拉力$P$为常数时,该问题的精确解为二次多项式\eqref{ch_introduction:ext}。所以首先选择二次多项式且满足位移边界条件$u(0)=0$的求解空间进行该问题的求解,此时位移$u$和虚位移$\delta u$假设为
\begin{subequations}\label{ch_energy:approx1}
\begin{align}
    u(x) &= a_1 x + a_2 x^2 \\
    \delta u(x) &= b_1 x + b_2 x^2
\end{align}
\end{subequations}
式中$a_i,i=1,2$为待定系数,而由于式\eqref{ch_energy:galerkin}对于所有求解空间中的虚位移$\delta u$都成立,所以$b_i,i=1,2$可以取任意实数。将假设的$u$和$\delta u$表达式\eqref{ch_energy:approx1}代入到里兹伽辽金问题\eqref{ch_energy:galerkin}中,方程等号左边为
\begin{equation}
\begin{split}
    \int_0^L \delta u_{,x} EA u_{,x} dx &= \int_0^L (b_1 + 2b_2 x)EA(a_1 + 2a_2 x) dx \\
    &= \begin{Bmatrix}
        b_1 \\ b_2
       \end{Bmatrix}^T
       \int_0^L EA \begin{bmatrix}
       1 & 2x \\ 2x & 4x^2 
       \end{bmatrix} dx
       \begin{Bmatrix}
        a_1 \\ a_2
       \end{Bmatrix} \\
    &= \begin{Bmatrix}
        b_1 \\ b_2
       \end{Bmatrix}^T
       EA\begin{bmatrix}
       L & L^2 \\ L^2 & \frac{4}{3}L^3 
       \end{bmatrix}
       \begin{Bmatrix}
        a_1 \\ a_2
       \end{Bmatrix} \\
    &= \boldsymbol b^T \boldsymbol K \boldsymbol a
\end{split}
\end{equation}
上式中$\boldsymbol a$、$\boldsymbol b$为系数向量。同样地,等式右边则改写为
\begin{equation}
\begin{split}
    \delta u P \vert_{x=L} + \int_0^L \delta u b dx &= (b_1 L + b_2 L^2)P + \int_0^L (b_1 x + b_2 x^2) b dx \\
                                                    &= (b_1 L + b_2 L^2)P + (\frac{b_1 L^2}{2} + \frac{b_2 L^3}{3}) b \\
                                                    &= (PL+\frac{b L^2}{2})b_1  + (PL^2 + \frac{b L^3}{3})b_2 \\
                                                    &= \begin{Bmatrix}
                                                   b_1 \\ b_2 
                                                    \end{Bmatrix}^T \begin{Bmatrix}
                                                   PL + \frac{bL^2}{2} \\ PL^2 + \frac{bL^3}{3}
                                                    \end{Bmatrix} \\
                                                    &= \boldsymbol b^T \boldsymbol f
\end{split}
\end{equation}
对式\eqref{ch_energy:galerkin}两端同时消去$\boldsymbol b^T$就可以确定待定系数向量$\boldsymbol a$为
\begin{equation}\label{ch_energy:coefficient}
    \boldsymbol{Ka} = \boldsymbol f \Rightarrow \boldsymbol a = \boldsymbol K^{-1}\boldsymbol f = \begin{Bmatrix}
    \frac{P+bL}{EA} \\ - \frac{b}{2EA}
    \end{Bmatrix}
\end{equation}
将系数$\boldsymbol a$表达式\eqref{ch_energy:coefficient}代入式\eqref{ch_energy:approx1}可得最终位移为
\begin{equation}\label{ch_energy:disp}
u(x) = -\frac{b}{2EA}x^2 + \frac{P+bL}{EA}x
\end{equation}\par
由求得的位移结果可以看出,求解得到的位移表达式\eqref{ch_energy:disp}与解析求得的位移表达式\eqref{ch_introduction:ext}相同,这是由于此时的求解空间包含精确解。在伽辽金法的使用过程中,由于大多数情况下并不知道精确解是什么函数,所以求解空间难以包含精确解。而在此问题中,势能泛函表达式\eqref{ch_energy:weak}中包含位移一阶导数,所以求解空间至少为线性函数(一阶导数不为零)。当求解空间为满足边界条件$u(0)=0$的线性函数时,假设的位移和虚位移表达式为
\begin{equation}
u(x) = a_1 x, \quad \delta u(x) = b_1 x
\end{equation}
以同样的步骤进行求解,可得相应的矩阵$\boldsymbol K$和$\boldsymbol f$为
\begin{equation}
    \boldsymbol K = [EAL], \; \boldsymbol f = \{PL+\frac{bL^2}{2}\}\;\Rightarrow \boldsymbol a = \{\frac{bL}{2EA} + \frac{P}{EA}\}
\end{equation}
此时求解得到的位移为
\begin{equation}\label{ch_energy:disp1}
u(x) = (\frac{bL}{2EA} + \frac{P}{EA})x
\end{equation}\par
图作出线性空间位移解\eqref{ch_energy:disp1}和二次空间的位移解\eqref{ch_energy:disp}的对比图。从图中可以看出,线性求解空间得到的位移接近二次空间的到的位移,并且在求解域处(杆件左右两端)位移相等。在伽辽金法中,虚位移$\delta u$也称为权函数(Weight function),而位移$u$称为试函数(Trial function)。如果试函数和权函数的求解空间包括精确解,此时的位移$u$即为精确解。如果试函数和权函数求解空间不包括精确解,位移$u$为求解空间到精确解最近的一点近似解。

\section{分段近似空间}
由上节可知,如果采用线性空间求解精确解为二次的问题可以得到一近似解,且解在求解域中具有插值的性质(求解域两端与精确解相等)。进一步如果采用分段多项式空间进行求解,能否类似得到一个具有插值性质的解。并且随着求解域尺寸的减小,近似解逐步收敛至精确解。\par
假设位移$u$及其变分$\delta u$在全域上为分段线性函数,且满足位移边界条件,即
\begin{subequations}\label{ch_energy:disph}
\begin{align}
u(x) =
\begin{cases}
    a_1 x & 0 \le x \le \frac{L}{2} \\
    a_2 + a_3 x & \frac{L}{2} < x \le L \\
\end{cases} \\
\delta u(x) =
\begin{cases}
    b_1 x & 0 \le x \le \frac{L}{2} \\
    b_2 + b_3 x & \frac{L}{2} < x \le L \\
\end{cases}
\end{align}
\end{subequations}\par
与上节相同,将假设的虚位移与位移代入到弱形式中,弱形式左端分别为
\begin{equation}\label{ch_energy:eq1}
\begin{split}
    \int_0^L \delta u_{,x}EA u_{,x} dx &= \int_0^{\frac{L}{2}} b_1 EA a_1 dx + \int_{\frac{L}{2}}^L b_3 EA a_3 dx \\
                                       &= b_1 \frac{EAL}{2} a_1 + b_3 \frac{EAL}{2} a_3 \\
                                       &= \begin{Bmatrix}
                                       b_1 \\ b_2 \\ b_3 
                                       \end{Bmatrix}^T
                                       \begin{bmatrix}
                                       \frac{EAL}{2} & 0 & 0 \\
                                       0 & 0 & 0 \\
                                       0 & 0 & \frac{EAL}{2}
                                       \end{bmatrix}
                                       \begin{Bmatrix}
                                       a_1 \\ a_2 \\ a_3 
                                       \end{Bmatrix} \\
                                       &= \boldsymbol b^T \boldsymbol K \boldsymbol a
\end{split}
\end{equation}
而等式右端为
\begin{equation}\label{ch_energy:eq2}
\begin{split}
    \delta u P + \int_0^L \delta u b dx &= (b_2+b_3L)P + \int_0^{\frac{L}{2}} b_1 x b dx + \int_{\frac{L}{2}}^L (b_2+b_3 x) b dx \\
                                        &= \frac{bL^2}{8}b_1 + (P+\frac{bL}{2})b_2 + (PL + \frac{3bL^2}{8})b_3 \\
                                        &= \begin{Bmatrix}
                                        b_1 \\ b_2 \\ b_3 
                                        \end{Bmatrix}^T
                                        \begin{Bmatrix}
                                        \frac{bL^2}{8} \\ P+\frac{bL}{2} \\ PL + \frac{3bL^2}{8}
                                        \end{Bmatrix} \\
                                        &= \boldsymbol b^T \boldsymbol f
\end{split}
\end{equation}\par
从式\eqref{ch_energy:eq1}中可以看出,矩阵$\boldsymbol K$不可逆,仅能确定系数$a_1$和$a_3$为
\begin{equation}
a_1 = \frac{bL}{4EA}, \quad a_3 = \frac{3bL}{4EA} + \frac{2P}{EA}
\end{equation}
此时的位移表达式为
\begin{equation}\label{ch_energy:eq3}
u(x) =
\begin{cases}
    \frac{bL}{4EA}x & 0 \le x \le \frac{L}{2} \\
    a_2 + (\frac{3bL}{4EA} + \frac{2P}{EA})x & \frac{L}{2} \le x \le L
\end{cases}
\end{equation}\par
将解得的位移表达式\eqref{ch_energy:eq3}对比精确解\eqref{ch_introduction:ext}可以看出,此时的求解得到的位移误差较大,且不连续。造成这种不连续的原因主要是近似的位移不连续造成的,即不满足\textbf{协调性}(Compatibility)。如果在位移表达式\eqref{ch_energy:disph}的基础上,进一步使假设的位移连续
\begin{equation}\label{ch_energy:coefficient1}
\begin{cases}
    \frac{L}{2}a_1 = a_2 + \frac{L}{2} a_3 \\
    \frac{L}{2}b_1 = b_2 + \frac{L}{2} b_3 
\end{cases} \Rightarrow
\begin{cases}
    a_1 = \frac{2}{L}a_2 + a_3 \\
    b_1 = \frac{2}{L}b_2 + b_3 
\end{cases}
\end{equation}
将关系式\eqref{ch_energy:coefficient1}代入到式\eqref{ch_energy:eq1}和\eqref{ch_energy:eq2}此时的弱形式左右两端为
\begin{equation}
\begin{split}
    \int_0^L \delta u_{,x}EA u_{,x} dx &= (\frac{2}{L}b_2+b_3) \frac{EAL}{2} (\frac{2}{L}a_2+a_3) + b_3 \frac{EAL}{2} a_3 \\
                                       &= b_2 \frac{2EA}{L} a_2 + b_2 EA a_3 + b_3 EA a_2 + b_3 EAL a_3 \\
                                       &= \begin{Bmatrix}
                                       b_2 \\ b_3 
                                       \end{Bmatrix}^T
                                       \begin{bmatrix}
                                       \frac{EAL}{2} & EA \\
                                       EA & EAL
                                       \end{bmatrix}
                                       \begin{Bmatrix}
                                       a_2 \\ a_3 
                                       \end{Bmatrix} \\
                                       &= \boldsymbol b^T \boldsymbol K \boldsymbol a
\end{split}
\end{equation}
\begin{equation}
\begin{split}
    \delta u P + \int_0^L \delta u b dx &= \frac{bL^2}{8}(\frac{2}{L}b_2 + b_3) + (P+\frac{bL}{2})b_2 + (PL + \frac{3bL^2}{8})b_3 \\
                                        &= (P+\frac{3bL}{4})b_2 + (PL + \frac{bL^2}{2})b_3 \\
                                        &= \begin{Bmatrix}
                                        b_2 \\ b_3 
                                        \end{Bmatrix}^T
                                        \begin{Bmatrix}
                                        P+\frac{3bL}{4} \\ PL + \frac{bL^2}{2}
                                        \end{Bmatrix} = \boldsymbol b^T \boldsymbol f
\end{split}
\end{equation}
此时可解得系数和相对应的位移为
\begin{equation}
\begin{cases}
a_2 = \frac{bL^2}{4EA} \\
a_3 = \frac{bL}{4EA} + \frac{P}{EA}
\end{cases} \Rightarrow
a_1 = \frac{2}{L}a_2 + a_3 = \frac{3bL}{4EA} + \frac{P}{EA}
\end{equation}
\begin{equation}
u(x) = \begin{cases}
(\frac{3bL}{4EA}+\frac{P}{EA})x & 0 \le x \le \frac{L}{2} \\
\frac{bL^2}{4EA} + (\frac{bL}{4EA}+\frac{P}{EA})x & \frac{L}{2} \le x \le L
\end{cases}
\end{equation} \par
对此近似节与精确解可以看出,由于满足协调性,位移误差相比于式\eqref{ch_energy:eq3}小得多,且在分段的端点处均满足插值性。但需要注意的是,求解的插值性在多维情况下难以实现,但近似空间的协调性依然是分段近似求解收敛性的保证。

